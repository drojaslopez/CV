\documentclass[letterpaper,11pt]{article}

\usepackage{latexsym}
\usepackage[empty]{fullpage}
\usepackage{titlesec}
\usepackage{marvosym}
\usepackage[usenames,dvipsnames]{color}
\usepackage{verbatim}
\usepackage{enumitem}
\usepackage[hidelinks]{hyperref}
\usepackage{fancyhdr}
\usepackage[spanish]{babel}
\usepackage{tabularx}
\input{glyphtounicode}

\pagestyle{fancy}
\fancyhf{} % Clear all header and footer fields
\fancyfoot{}
\renewcommand{\headrulewidth}{0pt}
\renewcommand{\footrulewidth}{0pt}

% Adjust margins
\addtolength{\oddsidemargin}{-0.5in}
\addtolength{\evensidemargin}{-0.5in}
\addtolength{\textwidth}{1in}
\addtolength{\topmargin}{-.5in}
\addtolength{\textheight}{1.0in}

\urlstyle{same}

\raggedbottom
\raggedright
\setlength{\tabcolsep}{0in}

% Sections formatting
\titleformat{\section}{
  \vspace{-4pt}\scshape\raggedright\large
}{}{0em}{}[\color{black}\titlerule \vspace{-5pt}]

% Ensure that generate PDF is machine readable/ATS parsable
\pdfgentounicode=1

%-------------------------
% Custom commands
\newcommand{\resumeItem}[2]{
  \item\small{
    \textbf{#1}{: #2 \vspace{-2pt}}
  }
}

% Just in case someone needs a heading that does not need to be in a list
\newcommand{\resumeHeading}[4]{
    \begin{tabular*}{0.99\textwidth}[t]{l@{\extracolsep{\fill}}r}
      \textbf{#1} & #2 \\
      \textit{\small#3} & \textit{\small #4} \\
    \end{tabular*}\vspace{-5pt}
}

\newcommand{\resumeSubheading}[4]{
  \vspace{-1pt}\item
    \begin{tabular*}{0.97\textwidth}[t]{l@{\extracolsep{\fill}}r}
      \textbf{#1} & #2 \\
      \textit{\small#3} & \textit{\small #4} \\
    \end{tabular*}\vspace{-5pt}
}

\newcommand{\resumeSubSubheading}[2]{
    \begin{tabular*}{0.97\textwidth}{l@{\extracolsep{\fill}}r}
      \textit{\small#1} & \textit{\small #2} \\
    \end{tabular*}\vspace{-5pt}
}

\newcommand{\resumeSubItem}[2]{\resumeItem{#1}{#2}\vspace{-4pt}}

\renewcommand{\labelitemii}{$\circ$}

\newcommand{\resumeSubHeadingListStart}{\begin{itemize}[leftmargin=*]}
\newcommand{\resumeSubHeadingListEnd}{\end{itemize}}
\newcommand{\resumeItemListStart}{\begin{itemize}}
\newcommand{\resumeItemListEnd}{\end{itemize}\vspace{-5pt}}

%-------------------------------------------
%%%%%%  CV STARTS HERE  %%%%%%%%%%%%%%%%%%%%%%%%%%%%


\begin{document}

%----------HEADING-----------------
\begin{tabular*}{\textwidth}{l@{\extracolsep{\fill}}r}
  \textbf{\href{-}{\Large Daniel Rojas Lopez}} & Email: \href{mailto:drojaslopez.ing@gmail.com}{drojaslopez.ing@gmail.com}\\
  Linkedin : \href{https://www.linkedin.com/in/drojaslopez/}{linkedin.com/in/drojaslopez/} & Celular: \href{tel:+56963660706}{+56963660706} \\
  GitHub : \href{https://www.github.com/drojaslopez}{github.com/drojaslopez} \\

\end{tabular*}

\section{Resumen Profesional}  
  \textit{Ingeniero Civil en Computación con 8 años de trayectoria en el diseño de arquitectura de software para sectores críticos. Especialista en la modernización de sistemas legados mediante microservicios y la integración de robustas capas de ciberseguridad en el ciclo de vida de desarrollo (DevSecOps), asegurando la escalabilidad y continuidad operativa en entornos de alta demanda.} 


%-----------Experiencia Laboral-----------------

\section{Experiencia Laboral}
  \resumeSubHeadingListStart
    \resumeSubheading
      {Banco de Chile}{Santiago, Chile}
      {Ingeniero SR de Software}{Enero 2024 -- Presente}
      \resumeItemListStart
        \resumeItem{Automatización para el Area de Soporte con IA}
          {Diseño e implementación de una solución automatizada con Agentes de IA (Copilot Studio y Power Platform) para la gestión de tickets de soporte, reduciendo el tiempo de respuesta promedio de 1 hora a 5 minutos.}
        \resumeItem{Ingeniero de Software}
          {Diseñé y lideré la implementación de la arquitectura para el área de banca-inversiones, logrando la integración exitosa de APIs externas y optimizando el procesamiento de datos en un 30\%}
        \resumeItem{Continuidad aplicativos de producción}
          {Lidero una célula de desarrollo de 4 ingenieros, definiendo los estándares técnicos y garantizando la disponibilidad del 99.9\% de los aplicativos críticos de producción.}
      \resumeItemListEnd
    \resumeSubheading
      {Banchile Inversiones}{Santiago, Chile}
      {Ingeniero SSR de Software}{Mar 2022 -- Dic 2023}
      \resumeItemListStart 
          \resumeItem{Automatización de Onboarding}
          {Diseño e implementación de una solución automatizada para el onboarding de colaboradores, reduciendo el tiempo de contratación de 3 meses a 1 semana. El sistema integró la solicitud de ingresos, creación de cuentas, entrega de equipos y configuración de entornos, eliminando errores manuales y optimizando la eficiencia operativa.}
          \resumeItem{Dashboard de monitoreo de opertividad de software y sistemas}
          {Implementación de soluciones de software para el área de inversiones, integrando sistemas internos y externos, mejorando la eficiencia y automatización de procesos.}
          \resumeItem{Responsable de celula de desarrollo}
          {Gestión integral de un equipo técnico (3 ingenieros y practicantes), abarcando la coordinación de flujos de trabajo, control de calidad y seguimiento riguroso de los plazos y objetivos del proyecto} 
      \resumeItemListEnd
    \resumeSubheading
      {Mobiquos}{Talca, Chile}
      {Ingeniero de Software y Proyectos}{Abr 2020 -- Feb 2022}
      \resumeItemListStart 
      \resumeItem{Proyecto IOT para ganaderia}
          {Desarrollo de una solución para optimizar el flujo de ingreso y control de ganado, logrando reducir en un 70\% el error humano y los tiempos de atención en la Feria de la Araucanía, además de mejorar la calidad de los datos recopilados en un 90\%.}
        \resumeItem{Lider de Proyectos}
          {Responsable equipo multidisciplinario enfocado en la implementación de soluciones a medida de la necesidad del cliente.}   
        \resumeItem{Ingeniero de Proyectos}
          {Responsable de proyectos de ingeniería e investigación, encargado del diseño e implementación de sistemas de acuerdo a los requerimiento de los clientes integrando dispositivo y aplicaciones (IoT). }
        
      \resumeItemListEnd
    \resumeSubheading
      {Mobiquos}{Talca, Chile}
      {Freelance}{Abr 2018 -- Mar 2020}
      \resumeItemListStart
        \resumeItem{Full Stack Developer}
          {Desarrollo de soluciones de software a medida para diversos clientes.}
        %\resumeItem{Manos Remotas}
        % {Responsable asistencia remota empresa Multinacional JTI Networks y Neutrona Networks (actualmente Flō Networks y enveedo)}
      \resumeItemListEnd      
  \resumeSubHeadingListEnd
  

%-----------Educación-----------------

\section{Educación}
  \resumeSubHeadingListStart
    \resumeSubheading
      {Universidad de Talca }{Talca, Chile}
      {Ingeniero Civil en Computación}{Mar 2010 -- Dic 2019}
    %\resumeSubheading
    % {Birla Institute of Technology and Science}{Pilani, India}
    % {Bachelor of Engineering in Electrical and Electronics; GPA: 3.66 (9.15/10.0)}{Aug 2008 -- July 2012}
  \resumeSubHeadingListEnd


%-----------Projects-----------------

\section{Proyectos}
  \resumeSubHeadingListStart
    \resumeSubItem{ERP - Sistema Integral de Gestión Empresarial}
      {Desarrollo de una plataforma ERP modular y altamente escalable, adaptada a requerimientos específicos del cliente. Implementada exitosamente en diversos sectores (clínicas dentales, gastronomía y retail) para optimizar el control de inventarios, gestión de ventas, administración de usuarios y generación de reportes de inteligencia de negocios (BI).}
      %{Stack de leguajes y herramientas utilizadas : Java, PostgreSQL, GitLab.}
    \resumeSubItem{App Móvil - Sistema de Control de Flota Vehicular}
      {Aplicación móvil diseñada para la gestión operativa en terreno de contratistas del sector eléctrico. Permite el monitoreo de vehículos, optimización de rutas y control logístico, mejorando la eficiencia en la asignación de recursos y tiempos de respuesta.}
      %{Stack de leguajes y herramientas utilizadas : Android SDK, Java, GitHub.}
    \resumeSubItem{Ciberseguridad OT - IDS/IPS para Redes Industriales(Proyecto de titulación)}
      {Sistema avanzado de detección y prevención de intrusos (IDS/IPS) para redes de operación (OT). Especializado en el análisis de tráfico del protocolo SCADA DNP3, diseñado para proteger infraestructuras críticas mediante la identificación de anomalías y ciberataques en tiempo real.}
      %{Stack de leguajes y herramientas utilizadas : Python, JavaScript, Bash, FreeBSD, Snort, Kali Linux, Metasploit.}
    \resumeSubItem{Web App - Plataforma de RRHH y Onboarding Digital}
      {Solución web integral para la digitalización de procesos de Recursos Humanos. Automatiza el ciclo de vida del empleado, desde la gestión documental y selección hasta el onboarding, facilitando la integración de talento y el cumplimiento administrativo.}
      %{Stack de leguajes y herramientas utilizadas : TypeScript, Node.js, React, MongoDB, Docker, Kubernetes, GitLab.}
    
  \resumeSubHeadingListEnd


%--------SKILLS------------

\section{Habilidades y Certificaciones }
  \resumeSubHeadingListStart
      \resumeSubItem{Lenguajes}{: Java, TypeScript, JavaScript, Python, SQL, Bash, PHP}
      \resumeSubItem{Base de Datos}{: Relacional (Oracle, PostgreSQL, MySql y Maria-db) y No Relacional(MongoDB)}
      \resumeSubItem{Cloud y Contenedores}{: GCP, OCI, AWS, Docker, Kubernetes}
      \resumeSubItem{Inglés}{: Intermedio B2, Capacidad para interpretar documentación técnica compleja, colaborar en foros especializados y mantener una comunicación fluida y efectiva en entornos internacionales.}
      \resumeSubItem{Certificados}{: Linux Foundation LFS151.x, OCI Foundations Associate, Desafio Latam(Backend con typescript, DEVOPS con IA)}
      \resumeSubItem{Metodologías}{: Agile, Scrum, Kanban}
      %\resumeSubItem{Patrones de diseño}{: Monolito, Clean Architecture o Hexagonal Architectur}
      
      \resumeSubHeadingListEnd


%-------------------------------------------

\end{document}
